\documentclass[11pt]{article}
\usepackage[a4paper, margin=1in]{geometry}
\usepackage{fontspec}
\usepackage{listings}

\title{Small step semantics for the STLC}
\author{Christian Kjær Larsen}

\usepackage{xcolor}

\definecolor{dkgreen}{rgb}{0,0.6,0}
\definecolor{gray}{rgb}{0.5,0.5,0.5}
\definecolor{mauve}{rgb}{0.58,0,0.82}

\lstdefinestyle{myScalastyle}{
  frame=tb,
  language=scala,
  aboveskip=3mm,
  belowskip=3mm,
  showstringspaces=false,
  columns=flexible,
  basicstyle={\small\ttfamily},
  numbers=none,
  numberstyle=\tiny\color{gray},
  keywordstyle=\color{blue},
  commentstyle=\color{dkgreen},
  stringstyle=\color{mauve},
  frame=single,
  breaklines=true,
  breakatwhitespace=true,
}

\lstset{style=myScalastyle}

\begin{document}
\maketitle
\section{Changes for adding $\mathbf{float}$}
We add $\mathbf{float}$ to the language by adding a new base type. We choose to overload $+$ and then only allow adding numbers of the same type. To add integers to floating point numbers we require explicit casts.

We start by extending the base syntax with decimal literals $m$ ($1.0$, $42.42$, $-3.14$, ...) with a decimal dot. Types and value are extended to be
\[
    \tau ::= ... \ |\ \mathbf{float}\ \quad
    v ::= ... \ |\ m\
\]
We add new syntactic constructs to the language for casts and floating point literals.
\[
    e ::= ... \ |\ m \ |\ \mathbf{float}(e)\ |\ \mathbf{int}(e)
\]
and we add new typing rules
\[
    \textsc{T-AddF}\frac{\Gamma \vdash e_1 : \mathbf{float}\quad \Gamma \vdash e_2 : \mathbf{float}}{\Gamma \vdash e_1 + e_2 : \mathbf{float}} \quad
    \textsc{T-Float}\frac{}{\Gamma \vdash m : \mathbf{float}}
\]
\[
    \textsc{T-ToFloat}\frac{\Gamma \vdash e : \mathbf{int}}{\Gamma \vdash \mathbf{float}(e) : \mathbf{float}} \quad
    \textsc{T-ToInt}\frac{\Gamma \vdash e : \mathbf{float}}{\Gamma \vdash \mathbf{int}(e) : \mathbf{int}}
\]
and new reduction rules
\[
    \textsc{AddF}\frac{}{m_1 + m_2 \rightarrow m}m = m_1 + m_2 \quad
    \textsc{Int}\frac{}{\mathbf{int}(m) \rightarrow n}n = \lfloor m \rfloor \quad
    \textsc{Float}\frac{}{\mathbf{float}(n) \rightarrow m}m = n
\]
And context rules
\[
    E ::= ...\ |\ \mathbf{int}(E) \ |\ \mathbf{float}(E)
\]
Now one writes $\mathbf{int}(\mathbf{float}(10) + 10.5)$ for adding two numbers of different types together and converting the result to an integer. We use the context rule to reduce under the $\mathbf{int}$.
\section{Implementation}
Included is a Scala 3 implementation of the lambda calculus with the above changes.

The abtract syntax is implemented using Scala 3 enums. A trait is used to encode the values. We have a type parameter A for annotations that are used for tracking source locations.
\begin{lstlisting}
enum Ty { ... }
type Name = String
sealed trait Value
enum Stlc[A] {
  case Lam(x: Name, t: Ty, body: Stlc[A], a: A) extends Stlc[A] with Value
  ...
}
\end{lstlisting}
In the program we can then use intersection types to require that a term is a value:
\begin{lstlisting}
def foo[A](v: Stlc[A] & Value): ...
\end{lstlisting}
\subsection{Type checking}
We include two functions. One for checking that a term has one of the required types and one for inferring the type of a term.
\begin{lstlisting}
type LStlc = Stlc[SourceLocation]
final case class TypeError(message: String, loc: SourceLocation)
def check(term: LStlc, ctx: Map[Name, Ty], ts: Set[Ty]): Either[TypeError, Ty]
def infer(term: LStlc, ctx: Map[Name, Ty]): Either[TypeError, Ty]
\end{lstlisting}
This is a trivial example of bidirectional type checking since we have annotated the lambda terms with the type of their arguments and therefore inference is always possible. The checking function in still nice for convenience though.
\subsection{Reduction}
We implement small step reduction with two functions. One for substituting a value for a variable, and one for taking a small step reduction. The stepping function will return nothing if we are stuck or if we have reached a value.
\begin{lstlisting}
def substValue[A](x: Name, e: Stlc[A], v: Stlc[A] & Value): Stlc[A]
def stepCBV[A](term: Stlc[A]): Option[Stlc[A]]
\end{lstlisting}
We have also included call-by-name semantics, but we will not discuss it here. The interpreter then works by reducing with the stepping function until we run out of gas or nothing is returned. This means that we are either stuck or have reached a value. If the typing rules are implemented correctly we should never be stuck.
\section{Running the interpreter}
To make the casts a bit more ergonomic and to avoid changing the parser we add them to the "standard library". This is done by type checking in a context
\begin{lstlisting}
val stdlib = Map(
  "int" -> Ty.Arrow(Ty.Float, Ty.Int),
  "float" -> Ty.Arrow(Ty.Int, Ty.Float)
)
\end{lstlisting}
and we have small step rules to reduce these functions to the $\eta$-expansion of the built-in casts.
\begin{lstlisting}
def stepCBV[A](term: Stlc[A]): Option[Stlc[A]] = {
  ...
  case Var("float", a) => Some(Lam("x", Ty.Int, ToFloat(Var("x", a), a), a))
  case Var("int", a) => Some(Lam("x", Ty.Float, ToInt(Var("x", a), a), a))
\end{lstlisting}
This makes sure that we can treat the casts as regular functions. How to run the project can be found in the README.md file included with the source code.
\end{document}
